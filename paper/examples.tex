\section{Motivating Example}
Consider the case of an epidemiologist tasked with an initial
exploratory analysis of data collected for individuals across
a battery of exams. In particular, she is interested in exploring
the relationship between income and the immune system,
controlling for factors such as weight and gender.

While careful modeling techniques will be necessary before putting
results into the final research paper, the epidemiologist is anxious
to get a quick and relatively accurate view into the data. However,
the data has been collected through surveys, namely CDC surveys (see Section~\ref{subsec:datasets} for details on such data),
and there is a significant amount of missing data across all
relevant attributes. The researcher must know develop a strategy
for filling in such missing values. She could drop any records with
missing values and continue her analysis with the remaining data,
or she could take advantage of existing correlations between attributes
to fit a more sophisticated model on the base tables, which may take
a long time. 

To complicate matters further, the epidemiologist realizes that
various of the steps she is interested in, including filtering and grouping,
are impacted differently by missing data, and that these step may change
repeatedly, as she wants to run various queries during the exploration phase.

Rather than work through
the implications of each case, she submits a query, as shown in Figure~\ref{fig:example-query}, to ImputeDB. The system, parameterized by a value indicating
a tradeoff between information loss and runtime cost, then finds
the optimal query plan, as shown in Figure~\ref{xxx}.

Tens of queries later, the epidemiologist has the holistic view of the
data that they required before carefully crafting their own tailored
imputation model. Knowing that there may be a need to explore
this data set further, they can easily incorporate their imputation model
into ImputeDB for future use.

\begin{figure}
\begin{lstlisting}[language=SQL]
SELECT
income,
AVG(white_blood_cell_ct)
FROM demo, exams, labs
WHERE 
gender = 2 AND weight >= 120 AND
demo.id = exams.id AND exams.id = labs.id
GROUP BY demo.income
\end{lstlisting}
\caption{A typical public health query on CDC's NHANES data}
\label{fig:example-query}
\end{figure}

\todo{add a small comment about the number of plans in the plancache
to highlight size of search space}
\todo{add a figure with the resulting query plan, and what its doing}
%%% Local Variables:
%%% mode: latex
%%% TeX-master: "main"
%%% End:
