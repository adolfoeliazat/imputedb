\section{Motivating Example}
An epidemiologist at the CDC is tasked with an 
exploratory analysis of data collected from individuals across
a battery of exams. In particular, she is interested in exploring
the relationship between income and the immune system,
controlling for factors such as weight and gender.

The epidemiologist is excited
to get a quick and accurate view into the data. However,
the data has been collected through CDC surveys (see~\Cref{subsec:datasets}),
and there is a significant amount of missing data across all
relevant attributes. 

Before she can perform her queries, the epidemiologist must develop a strategy
for handling missing values. She currently has two options:
1) she could drop records with missing values in relevant fields,
2) she could use a standard data imputation package on all of her data. Both of these
approaches have significant drawbacks. For the query pictured in~\Cref{fig:example-query}, 
(1) drops 1492 potentially relevant tuples,
while her previous experiences with (2) have proven to be expensive. The epidemiologist needs a more complete picture
of the data, so (1) is insufficient, and she is under time pressure, so (2)
is infeasible.

She can run her queries immediately and finish her report if
she uses \ProjectName{}, which allows her to write her
query in standard SQL and automatically performs the imputations necessary to fill in the missing data.
An example query is shown in~\Cref{fig:example-query}.

\begin{figure}
\begin{lstlisting}[language=SQL]
SELECT income, AVG(white_blood_cell_ct)
FROM demo, exams, labs
WHERE gender = 2 AND 
      weight >= 120 AND
      demo.id = exams.id AND 
      exams.id = labs.id
GROUP BY demo.income
\end{lstlisting}
\caption{A typical public health query on CDC's NHANES data.}
\label{fig:example-query}
\end{figure}

\begin{figure}
  \Tree
  [.$\pi_{\text{income, AVG(white\_blood\_cell\_ct)}}$
    [.$g_{\text{income, AVG(white\_blood\_cell\_ct)}}$
      [.\colorbox{pink}{$\mu_{\text{demo.income}}$}
        [.$\bowtie_{\text{exams.id} = \text{demo.id}}$
          [.\colorbox{pink}{$\mu_{\text{labs.white\_blood\_cell\_ct}}$}
            [.$\bowtie_{\text{exams.id} = \text{labs.id}}$
              [.$\sigma_{\text{exams.weight} \geq 120}$ 
                [.\colorbox{pink}{$\mu_{\text{exams.weight}}$} exams ] 
              ] 
              labs 
            ]
          ]
        [.$\sigma_{\text{demo.gender} = 2}$ demo ]
      ] 
    ] 
  ] 
  ]
\vspace{0.5\baselineskip}
\caption{A quality-optimized plan for the query in~\Cref{fig:example-query}. The operators $\sigma$, $\pi$, $\bowtie$, and $g$ are the standard relational selection, projection, join, and group-by/aggregate, $\mu$ and $\delta$ are specialized imputation operators (\Cref{sec:operators}).}
\label{fig:quality-plan}
\end{figure}

\subsection{Planning with \ProjectName{}}
The search space contains plans of varying performance and quality of results, as a product of the multiple possible locations for imputation.
The user can influence the final plan selected by \ProjectName{}
through a trade-off parameter $\alpha \in [0, 1]$, where low $\alpha$ 
prioritizes the quality of the query results and high $\alpha$ prioritizes performance.

For the query in~\Cref{fig:example-query}, \ProjectName{} generates
the plans in~\Cref{fig:quality-plan} and~\Cref{fig:fast-plan}, when 
optimizing for quality and performance, respectively.

\Cref{fig:quality-plan} shows a quality-optimized plan that uses the impute operation $\mu$, which employs an iterative regression model to fill in missing values rather than dropping tuples.
It waits to impute \verb|demo.income| until after the final join has taken place, but other imputations take place earlier on in the plan, some before filtering and join operations.
Imputations are placed to maximize ImputeDB's estimate of the quality of the overall results.
Meanwhile, \Cref{fig:fast-plan}'s performance-optimized plan uses the drop operation $\delta$ instead of filling in missing values.
However, it is a significant improvement over dropping all tuples with missing data, as it only drops tuples with missing values in attributes which are referenced by the rest of the plan. In both cases, only performing
imputation on the necessary data yields a query that is much faster
than performing imputation on the whole dataset and then executing
the query.



\begin{figure}
\Tree
  [.$\pi_{\text{income, AVG(white\_blood\_cell\_ct)}}$
    [.$g_{\text{income, AVG(white\_blood\_cell\_ct)}}$
      [.\colorbox{pink}{$\delta_{\text{demo.income, labs.white\_blood\_cell\_ct}}$}
        [.$\bowtie_{\text{exams.id} = \text{labs.id}}$
          [.$\bowtie_{\text{demo.id} = \text{exams.id}}$
            [.$\sigma_{\text{demo.gender} = 2}$ demo ]
            [.$\sigma_{\text{exams.weight} \geq 120}$ [.\colorbox{pink}{$\delta_{\text{exams.weight}}$} exams ] ] ] labs ] ] ] ]
\caption{A performance-optimized plan for the query in~\Cref{fig:example-query}.The operators $\sigma$, $\pi$, $\bowtie$, and $g$ are the standard relational selection, projection, join, and group-by/aggregate, $\mu$ and $\delta$ are specialized imputation operators (\Cref{sec:operators}).}
\label{fig:fast-plan}
\end{figure}

\subsection{\ProjectName{} Workflow}
Now that the epidemiologist has \ProjectName{}, she can explore
the dataset using SQL. For each query, she repeatedly chooses a value
for $\alpha$ and runs the query until she is unsatisfied with the runtime.
Rather than commit \emph{a priori} to a runtime bound for her query,
which may be unsatisfiable, this iterative approach gives her 
immediate feedback.

Tens of queries later, our epidemiologist has a holistic view of the data and she has the information that she needs to construct a tailored imputation model for her queries of interest.
Knowing that there may be a need to explore this data set further, she can easily incorporate her imputation model into ImputeDB for future use.

%This approach allows her to maximize the query result quality, while
%maintaining an acceptable runtime.
%
%%She indicates her preference for performance or for accuracy by providing a configuration parameter $\alpha$ to ImputeDB in addition to her query.
%%$\alpha$ is a dimensionless parameter in $[0,1]$, and it indicates the relative importance of the quality of query results and of query performance.
%So, a high $\alpha$ results in a fast query plan that may compromise accuracy, while a low $\alpha$ results in a high quality query plan that may take longer to run.
%In our intended workflow, the analyst picks a value of $\alpha$ and incrementally increases it to increase query result quality until the query takes too long to run.
%Fig.~\ref{fig:quality-plan} and Fig.~\ref{fig:fast-plan} show plans for the example query in Fig.~\ref{fig:example-query} when optimizing for quality and performance, respectively.

%1 - How great life is with the project
% - intended work flow



%%% Local Variables:
%%% mode: latex
%%% TeX-master: "main"
%%% End:
