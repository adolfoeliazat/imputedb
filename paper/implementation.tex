\section{Implementation}
Adding imputation and our optimization algorithm to a standard SQL database requires modifications to several key database components.
In this section we discuss the changes which must be made to implement dynamic imputation.

\begin{itemize}
\item \textbf{Extended histograms:}
  Ranking query plans with imputation requires estimates of the distribution of missing values in the base tables and in the outputs of sub-plans.
  We extended standard histograms to include the count of missing values for each attribute.
  On the base tables, this provides an exact count of the missing values.
  In sub-plans we use simple heuristics---as discussed in Section~\ref{sec:cardinal}---to estimate the number of missing values after applying both standard and imputation operators.
  These cardinality estimates are critical to the optimizer's ability to compare the performance and imputation quality of different plans.

\item \textbf{Dirty sets:}
  The planner needs to know which attributes in the output of a plan might contain missing values so it can insert the correct imputation operators.
  To provide the planner with this information, we over-approximate the set of attributes in a plan that may have missing values.
  Each plan has an associated \emph{dirty set}, described in Section~\ref{sec:placement}, which tracks these attributes.

\item \textbf{Imputation operators:}
  We extend the set of logical operations available to the planner with the \emph{Impute} and \emph{Drop} operators.
  The database must have implementations for both operators and be able to correctly place them while planning.
  In addition to the normal heuristics for estimating query runtime, these operators have a cost function $\textsc{Penalty}$ (Section~\ref{sec:quality}) which estimates the quality of their output.
  The planner must be able to optimize both cost functions and select an appropriate plan from the set of Pareto-optimal plans.
  
  % While the \emph{Impute} and \emph{Drop} operations are not exposed directly to the user, they are a core extension the the standard logical operator nodes in the planner.
  % Both operate on a subset of attributes, determined by a combination of the dirty set and planner-provided attributes. These are used to enumerate local imputation alternatives.
  % Both operators can compute cardinality and adjust histogram estimates.
  % In contrast to standard operators, they also provide $\textsc{Penalty}$ and $\textsc{Time}$ estimates (\Cref{sec:cost-model}), based on the underlying imputation algorithm and the data input.

\item \textbf{Integrated optimizer:}
  We believe that it would be relatively simple to extend an existing query optimizer to insert imputations immediately after scanning the base tables.
  However, integrating the imputation placement rules into the optimizer allows us to explore a large space of query plans which contain imputations.
  In particular, this tight integration allows us to jointly choose the most effective join order and imputation placement.
  Our goal of integrating imputation placement into the optimizer requires the other modifications that we have discussed.
\end{itemize}

%%% Local Variables:
%%% mode: latex
%%% TeX-master: "main"
%%% End:
