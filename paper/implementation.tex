\section{Implementation}
We discuss the main components required to implement dynamic query imputation.

\textbf{Extended histograms}: Planning imputation requires estimates of the distribution of missing values. 
We extended standard histograms to include the count of missing values across table attributes.
On the base tables, this provides an exact count of the missing fields. 
In sub-plans, it allows us to use simple heuristics to estimate the number of missing values after applying
both standard and imputation operators. These cardinality estimates are critical to the optimizer.

\textbf{Dirty sets}: We over-approximate the attributes in a plan that may have missing values and need to be
imputed (using $\mu$ or $\delta$). We extended
the logical nodes in our planning data structures to carry a set of fully qualified attribute names as the dirty set. 

\textbf{Additional Logical Operators}: While the $\mu$ and $\delta$ operations are not exposed directly to the user, they are a core extension
the the standard logical operator nodes in the planner.
Both operate on a subset of attributes, determined by
a combination of the dirty set and planner-provided attributes. These are used to enumerate local imputation alternatives. Both operators
can compute cardinality and adjust histogram estimates. In contrast to standard operators, they also provide $\textsc{Penalty}$ and $\textsc{Time}$ estimates (\Cref{sec:cost-model}),
based on the underlying imputation algorithm and the data input.

\textbf{Core optimizer}: Imputations early in the plan (e.g. before a selection) could be added to existing optimizers. However, the full
planning algorithm used in \ProjectName{} is tightly integrated with core parts of the optimizer, such as join enumeration. This tight integration allows \ProjectName{}
to influence join and imputation order, through the imputation cost model, and effectively prune out plans that are not in the relevant Pareto set.
Substituting the standard optimizer for a custom version tailored to dynamic imputation optimization is critical to obtain the
performance provided by \ProjectName{}.


%%% Local Variables:
%%% mode: latex
%%% TeX-master: "main"
%%% End:
