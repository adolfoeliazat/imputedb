\section{Related Work}

There is related work in three primary areas: statistics, database research, and forecasting.

\subsection{Missing Values and Statistics}

Imputation of missing values is widely studied in the statistics and machine
learning communities. Missing data
can appear for a variety of reasons~\cite{gelman2006data}.
It can be missing completely at random or conditioned on
existing values (observed and missing). Methods in the statistical community
focus on correctly modeling relationships between attributes to account for different forms
of missingness.
For example, \textcite*{burgette2010multiple} use iterative decision trees for imputing missing data.

The computational difficulties of imputing on large base
tables are well-known and can limit approaches. % \citeauthor{akande2015empirical} analyze the performance of various
% multiple imputation techniques on the American Community Survey dataset~\cite{akande2015empirical}.
For example, \textcite*{akande2015empirical} find that one approach (MI-GLM) is prohibitively expensive when attempting to impute on the American Community Survey dataset.
In this case, the complexity comes from the large domains of the variables in the ACS data (up to ten categories in their case).
In contrast, \ProjectName{} allows users to specify a trade-off between
imputation quality and runtime performance, allowing users to perform queries
directly on datasets which are too large to fully impute quickly. Furthermore,
the query planner's imputation is guided by the requirements of each specific
query's operators, rather than requiring broad assumptions about query
workloads.  

\subsection{Missing Values and Databases}
There is a long history in the database community surrounding the
treatment of \nullv{}.
Multiple papers have described various (and at times conflicting) treatments of \nullv{}s~\cite{codd1973understanding,grant1977null}.
\ProjectName's design invariants eliminate
the need to handle \nullv{} value semantics, while guaranteeing query evaluation soundness.

Database system developers and others have worked on techniques to automatically
detect dirty values, whether missing or otherwise, and rectify the errors if
possible. \textcite*{hellerstein2008quantitative} surveys the methods and systems related to missing value detection and correction.

\textcite*{wolf2007query} directly treat queries over databases with missing values using a statistical approach, taking advantage of correlations between
attributes. The tuples that match the original query as well as a ranking of
tuples with incomplete data that may match are returned. Our work differs in
that we allow any well-formed statistical technique to be used for imputation
and focus on returning results to the analyst as if the database had been
complete.

Designers of data stream processing systems frequently confront missing values
and consider them carefully in query processing. Often, if a physical process
like sensor error is the cause of missing values, values can be imputed with
high confidence. \textcite*{fernandez2009inter} use feedback punctuation to
dynamically
reconfigure the query plan for state-dependent
optimizations as data continues to arrive. One of the applications of this framework is to avoid
expensive imputations as real-time conditions change. 
%We prove our
%algorithm and cost model produces an optimal set of plans to choose from,
%given the constraints of the search space.

Other work has looked at integrating statistical models into databases.
For example, BayesDB~\cite{mansinghka2015bayesdb} provides users with a simple interface to leverage statistical inference techniques in a database. Non-experts
can use a simple declarative language (an extension of SQL), to specify models
which allow missing value imputation, among other broader functionality.
Experts can further customize strategies and incorporate domain knowledge to
improve performance and accuracy.
While BayesDB can be used for value imputation, this step is not framed
within the context of query planning, but rather as an explicit statistical
inference step within the query language, using the \verb|INFER| operation. 
BayesDB provides a great alternative for bridging the gap between
traditional databases and sophisticated modeling software. \ProjectName{}, in
contrast, aims to remain squarely in the database realm, while allowing
users to directly express queries on a potentially larger subset of their data.

\review{
\textcite*{yang2015lenses} develop an incremental approach to ETL data cleaning, through the use
of probabilistic compositional repair operators, called \emph{Lenses}, which balance quality of cleaning
and cost of those operations. These operators can be used
to repair missing values by imposing data constraints such as \texttt{NOT NULL}. However, the model
used to impute missing values must be pre-trained and all repairs are done on a per-tuple basis
rather than at the plan-level.
}

\ProjectName's cost-based query planner 
is partially based on the seminal work developed for System R's query planning~\cite{blasgen1981system}.
However, in contrast to System R, \ProjectName{} performs additional optimizations for imputing missing data and
uses histogram transformations to account for the way relational and imputation operators affect the underlying
tuple distributions.

\review{
The planning algorithm that we present has similarities to work on multi-objective query optimization~\cite{trummer2014}.
Both algorithms handle plans which have multiple cost functions by maintaining sets of Pareto optimal plans and pruning plans which become dominated.
In \ProjectName{}, costs are expected to grow monotonically as the size of the plan increases, which simplifies the optimization problem significantly.
\textcite*{trummer2014} handle the more complex case of cost functions which are piecewise linear, but not necessarily monotonic.
}

\subsection{Forecasting and Databases}
\textcite*{parisi2011embedding} introduce the idea of incorporating time-series forecast operators into
databases, along with the necessary relational algebra extensions. Their work explores the theoretical
properties of forecast operators and generalizes them into a family of operators, distinguished by
the type of predictions returned. They highlight the use of forecasting for replacing missing values.
In subsequent work~\cite{parisi2013temporal}, they identify various equivalence and containment
relationships when using forecast operators, which could be used to perform query plan transformations that guarantee the same result. They
explore forecast-first and forecast-last plans, which perform forecasting operations before and after executing all traditional
relational operators, respectively. 

\textcite*{fischer2013towards} describe the architecture of a DBMS with integrated forecasting operations for time-series, detailing
the abstractions necessary to do so.
In contrast to this work, \ProjectName{} is targeted at generic value imputation, and is not tailored to time-series.
The optimizer is not based on equivalence transformations, nor are there guarantees of equal
results under different conditions. Instead, the optimizer allows users to pick their trade-off between
runtime cost and imputation quality. The search space considered by our optimizer is broader, not just
forecast-first/forecast-last plans, but rather imputation operators can be placed anywhere in the query plan
(with some restrictions). The novelty of our contribution lies in the successful incorporation of
imputation operations in non-trivial query plans with cost-based optimization.

\textcite*{duan2007processing} describe the Fa system and its declarative language for time-series forecasting. Their
system automatically searches the space of attribute combinations/transformations and statistical models
to produce forecasts within a given accuracy threshold. Accuracy estimates are determined using
standard techniques, such as cross-validation. 
As with Fa, \ProjectName{} provides a declarative language, as
a subset of standard SQL.\@ \ProjectName{}, however, is not searching the space of possible
imputation models, but rather the space of query plans that incorporate imputation operators. Another major point
of distinction between \ProjectName{} and Fa is that the latter doesn't consider trade-offs between accuracy and computation time, but rather returns the most accurate forecast (with some stopping criterion).



%%% Local Variables:
%%% mode: latex
%%% TeX-master: "main"
%%% End:
