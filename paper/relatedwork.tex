\section{Related Work}

\subsection{Missing Values and Statistics}

Imputation of missing values is a widely studied field within the statistics and machine
learning communities. As highlighted in \cite{gelman2006data}, missing data
can appear for a variety of reasons, including both random and conditioned on
existing values (observed and missing). Methods in the statistical community
focus on correctly modeling relationships between the attributes to factor in
varied forms of missingness. For example, Burgette and Reiter
(\cite{burgette2010multiple}) discuss the usage of sequential regression trees
for imputing missing data.

In \cite{akande2015empirical}, Akande et al analyze the performance of various
multiple imputation techniques on the American Community Survey dataset
(Sec~\ref{subsec:datasets}). The computational difficulties of imputing on large base
tables are well known and can limit approaches. For example, Akande finds that
one approaches (MI-GLM) crashes when attempting to impute on data that
includes variables with potentially large domains (ten categories specifically).
In contrast, \ProjectName{} allows users to specify a tradeoff between
information loss and computational complexity (in terms of time). Furthermore,
the query planner's imputation is guided by the requirements of each specific
query's operators, rather than requiring broad assumptions about query
workloads.  

\subsection{Missing Values and Databases}
There is a long history in the database community surrounding the
treatment of nulls. As early as 1973, \cite{codd1973understanding}
provides a treatment of the semantics of null. Multiple
papers have described various (at times conflicting) treatments
of nulls \cite{grant1977null}. \ProjectName's main design invariant - that no relational operator
see attributes with missing values for attributes it must operate on and users should never see
missing data - eliminates
the need to handle null value semantics, while guaranteeing soundness (modulo
imputation strategy) of the query evaluation.

Database system developers and others have worked on techniques to automatically
detect dirty values, whether missing or otherwise, and rectify the errors is
possible. A survey of methods and systems is provided in
\cite{hellerstein2008quantitative}.

BayesDB \cite{mansinghka2015bayesdb} provides users with a simple interface to 
leverage statistical inference techniques on a database. Non-experts
can use a simple declarative language, akin to SQL, to specify models
which allow missing value imputation, amongst other broader functionality.
Experts can further customize strategies and express domain knowledge to
improve performance and accuracy.

While BayesDB can be used for value imputation, this step is not framed
within the context of query planning, but rather as an explicit step, using the
\verb|INFER| operation. 

BayesDB provides a great alternative for bridging the gap between
traditional databases and sophisticated modeling software. \ProjectName{}, in
contrast, aims to remain squarely in the database realm, while allowing
users to perform queries on a potentially larger subset of their data.

\ProjectName's cost-based query planner 
is partially based on the seminal work developed for System R's query planning\cite{blasgen1981system}.
However, in contrast to System R, \ProjectName{} performs additional histogram transformations to account
for the changing nature of missing values.





%%% Local Variables:
%%% mode: latex
%%% TeX-master: "main"
%%% End:
