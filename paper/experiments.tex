\section{Experiments}

To evaluate the performance of our system, we generated plans for queries on two data sets: the American Community Survey and a synthetic dataset.
\subsection{Data sets} \label{subsec:datasets}

The American Community Survey (ACS) provides a number of public data sets collected by the U.S. Census Bureau.
We used a cleaned version of the 2012 Public Use Microdata Sample (PUMS) data kindly provided by the authors of~\cite{akande2015empirical}.
The cleaning procedure is described in detail in~\cite{akande2015empirical}, but to summarize, the following data were removed:
\begin{itemize}
\item Rows corresponding to vacant houses or single occupant households.
\item Identification variables (e.g. area codes, serial numbers).
\item Flag variables.
\item Continuous variables.
\end{itemize}
The final data set consists of a single table with 671,153 rows and 37 columns, where all column entries are integers.

In order to evaluate our implementation, we created a dirtied variant of the ACS data by randomly eliminating
10\% of the fields across the table.

In order to evaluate additional queries, we constructed a synthetic table, consisting of 10 columns and 10,000 rows, with
all values drawn from a uniform distribution between 0 and 100. To create a dirtied variant of the synthetic data, we randomly deleted 30\% of the fields
across the table.

\subsection{Queries}

We collected a set of queries (Figure~\ref{fig:queries}) that we think are both representative, in that they could reasonably be written by a user in the course of data analysis, and interesting to plan.

The queries on the ACS data all consist only of selections, filters, and aggregations.
The ACS data is contained in a single table, so there are relatively few join queries that are interesting on this data set and remain relevant for analysts, rather
than contrived solely for the purpose of experimentation.
However, even with select, filter, and aggregate, there are interesting choices to make with imputation placement. In order to explore joins, we leverage the synthetic
dataset and write various ad-hoc queries.

\todo{fix query references}
\begin{table}
  \centerfloat
  \newcounter{queryno}
\begin{tabular}{cl}
\toprule
\# & \multicolumn{1}{c}{Queries on CDC data} \\
\midrule
1 & 
\begin{minipage}{6in}
\begin{lstlisting}[breaklines]
SELECT income, AVG(cuff_size) FROM demo, exams 
WHERE demo.id = exams.id AND height >= 150 GROUP BY income;
\end{lstlisting}
\end{minipage}\refstepcounter{queryno}\label[query]{q1} \\
2 & 
\begin{minipage}{6in}
\begin{lstlisting}[breaklines]
SELECT income, AVG(creatine) FROM demo, exams, labs 
WHERE demo.id = exams.id AND exams.id = labs.id AND income >= 13 AND income <= 15 AND weight >= 63 GROUP BY income;
\end{lstlisting}
\end{minipage}
\refstepcounter{queryno}\label[query]{q2} \\
3 & 
\begin{minipage}{6in}
\begin{lstlisting}[breaklines]
SELECT AVG(blood_lead) FROM demo, exams, labs 
WHERE demo.id = labs.id AND labs.id = exams.id AND age_yrs <= 6;
\end{lstlisting}
\end{minipage}\refstepcounter{queryno}\label[query]{q3}\\
4 & 
\begin{minipage}{6in}
\begin{lstlisting}[breaklines]
SELECT gender, AVG(blood_pressure_systolic) FROM demo, labs, exams 
WHERE demo.id = labs.id AND labs.id = exams.id AND body_mass_index >= 30 GROUP BY gender;
\end{lstlisting}
\end{minipage}\refstepcounter{queryno}\label[query]{q4}\\
5 & 
\begin{minipage}{6in}
\begin{lstlisting}[breaklines]
SELECT AVG(waist_circumference) FROM demo, exams 
WHERE demo.id = exams.id AND height >= 150 AND weight >= 100;
\end{lstlisting}
\end{minipage}\refstepcounter{queryno}\label[query]{q5}\\
  \midrule
\# & \multicolumn{1}{c}{Queries on freeCodeCamp data} \\
\midrule
6 & 
\begin{minipage}{6in}
\begin{lstlisting}[breaklines]
SELECT attendedbootcamp, AVG(income) FROM fcc
WHERE income >= 50000 GROUP BY attendedbootcamp;
\end{lstlisting}
\end{minipage}\refstepcounter{queryno} \label[query]{q6} \\
7 & 
\begin{minipage}{6in}
\begin{lstlisting}[breaklines]
SELECT AVG(commutetime) FROM fcc
WHERE gender = "female" AND countrycitizen = "United States";
\end{lstlisting}
\end{minipage}\refstepcounter{queryno} \label[query]{q7} \\
8 & 
\begin{minipage}{6in}
\begin{lstlisting}[breaklines]
SELECT schooldegree, AVG(studentdebtowe) FROM fcc
WHERE studentdebtowe > 0 GROUP BY schooldegree;
\end{lstlisting}
\end{minipage}\refstepcounter{queryno} \label[query]{q8}\\
9 & 
\begin{minipage}{6in}
\begin{lstlisting}[breaklines]
SELECT attendedbootcamp, AVG(gdp_per_capita) FROM fcc, gdp
WHERE fcc.countrycitizen = gdp.country AND age >= 18
GROUP BY attendedbootcamp;
\end{lstlisting}
\end{minipage}\refstepcounter{queryno} \label[query]{q9}\\
\bottomrule
\end{tabular}

%%% Local Variables:
%%% mode: latex
%%% TeX-master: "main"
%%% End:

  \caption{The queries used in our experiments.}
  \label{fig:queries}
\end{table}

\begin{table}
  \centerfloat
  \begin{tabular}{cSSSSSSS}
\toprule
\multicolumn{2}{c}{} & \multicolumn{2}{c}{Imputed ($\alpha=0.0$)} & \multicolumn{2}{c}{Imputed ($\alpha=1.0$)} \\
\cmidrule(r){3-4}
\cmidrule(l){5-6}
\# & \multicolumn{1}{c}{Base error} & \multicolumn{1}{c}{Error} & \multicolumn{1}{c}{Time (s)} & \multicolumn{1}{c}{Error} & \multicolumn{1}{c}{Time (s)} \\
\midrule
\ref{q1} & 5.25e+03 & 0.00e+00 & 1.723 & -3.60e+03 & 11.008 \\
\ref{q2} & 3.38e+04 & 0.00e+00 & 1.748 & -3.37e+04 & 10.144 \\
\ref{q3} & 9.32e-04 & 0.00e+00 & 1.732 & 2.54e-02 & 33.04 \\
\ref{q4} & 1.78e+05 & 0.00e+00 & 1.778 & -1.55e+05 & 21.013 \\
\ref{q5} & 9.89e-04 & 0.00e+00 & 1.712 & 9.89e-02 & 7.981 \\
\ref{q6} & 1.01e-02 & 0.00e+00 & 1.982 & 1.13e-02 & 52.309 \\
\ref{q7} & 0.00e+00 & 0.00e+00 & 43M49.786 & 0.00e+00 & 5.946 \\
\ref{q8} & 0.00e+00 & 0.00e+00 & 1.776 & 0.00e+00 & 3.172 \\
\ref{q9} & \multicolumn{1}{c}{--} & \multicolumn{1}{c}{--} & 1.919 & \multicolumn{1}{c}{--} & 2M40.384 \\
\ref{q10} & \multicolumn{1}{c}{--} & \multicolumn{1}{c}{--} & 0.01 & \multicolumn{1}{c}{--} & 7.786 \\
\ref{q11} & \multicolumn{1}{c}{--} & \multicolumn{1}{c}{--} & 0.007 & \multicolumn{1}{c}{--} & 0.103 \\
\ref{q12} & 0.00e+00 & 0.00e+00 & 0.008 & 1.00e-02 & 0.175 \\
\ref{q13} & \multicolumn{1}{c}{--} & \multicolumn{1}{c}{--} & 0.407 & \multicolumn{1}{c}{--} & 0.4 \\
\ref{q14} & 0.00e+00 & 0.00e+00 & 0.752 & 4.37e-01 & 1M14.221 \\
\bottomrule
\end{tabular}

  \caption{Running time and error for queries with different imputation levels.}
  \label{fig:experiments}
\end{table}

Figure~\ref{fig:experiments} provides a summary of our experimental results. For purposes of simplicity, we present only the results associated with
queries that return a result size matching our base case. This addresses the fact that deriving a reasonable error metric for set-queries 
is inherently task specific (e.g. consider tuning precision vs. recall). Meanwhile, for all queries that match, we apply a simple sum of squared errors
(SSE). Figure~\ref{fig:experiments} indicates the query number and the error produced by a query engine with traditional SQL null semantics (base error).
For imputations, we present two extremes of the $\alpha$ parameter: 0 and 1, which reflect a a focus on runtime and information loss, respectively.
For each query, the table displays the new error ($\epsilon$), the planning time ($t_p$) and the running time ($t_r$) in seconds.

The overall low planning times highlight the practicality of our algorithm, despite the exponential complexity. Runtimes clearly vary based on $\alpha$, allowing
users to tune their query performance as desired.

\todo{add result-specific analysis}

Finally, we performed $CE-CART$ imputation on the entirety of the ACS table, in order to compare our system to the traditional, pre-processing approach to imputation.
We note that the latter took \todo{add time}, compared to our maximum of \todo{add time}. Given the varying needs of database users, \ProjectName{} clearly allows flexibility without
a significant performance hit.


%%% Local Variables:
%%% mode: latex
%%% TeX-master: "main"
%%% End:
