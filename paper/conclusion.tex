\section{Conclusion}
Our evaluation of \ProjectName{} shows that missing values and their imputation can be successfully integrated into the relational calculus and
existing cost-based plan optimization frameworks. We implement imputation actions, such as dropping or imputing values with a machine
learning technique, as operators in the algebra and use a simple, but effective, cost model to consider tradeoffs in 
imputation quality and runtime. We showed that different values for the tradeoff parameter can yield substantially
different plans, allowing the user to express their preferences for performance on a per-query basis.
Our experiments with \textit{CDC}, \textit{freeCodeCamp} and \textit{ACS} survey-based datasets
showed that \ProjectName{} can be used successfully with real-world data, expressing standard SQL queries over data with missing values. 
We craft a series of realistic queries, meant to resemble the kind of questions an analyst might ask during
the data exploration phase. The plans selected for each query execute an order-of-magnitude faster than
 the standard approach of imputing on the entire base tables and
then formulating queries. Furthermore, the difference in query results between the two approaches is
shown to be small in all queries considered\todo{quantify this}. Data analysts need not commit to a specific imputation strategy and can instead
vary this across queries.

We highlight the long history of dealing with missing data both in the statistical learning and database communities.
Similarly to existing work, we consider the impact of null values in databases and develop a simple set of invariants to 
successfully plan around them. In contrast to existing statistical learning work, our emphasis is not on the specific algorithm
used to impute but rather on the placement of imputation steps in the execution of a query. In contrast to existing database work,
we incorporate imputation into a cost-based optimizer and hide any details
regarding missing values inside the system, allowing users to use traditional SQL and engage in normal workloads.
While prior work has incorporated operations such as prediction into their databases, this has been domain-specific
in some cases, and in others, they have not integrated these operations into the planner. The novelty
of our contribution lies in the formalization of missing value imputation for query planning, which results in performance
gains over traditional approaches. This approach acknowledges that different users performing different queries on the same
dataset will likely have varying imputation needs and that execution plans should appropriately reflect that variation.

\subsection{Future Work}
\ProjectName{} opens up multiple avenues for further work. We sketch out some of these
directions.

\textbf{Imputation confidence:} Obtaining a confidence measure for query results
produced when missing values are imputed with a machine learning model is
another possible improvement to the system. In most cases, the data imputed
is unlikely to follow a standard distribution, which implies that any confidence
estimates will result from iterative approaches such as cross-validation\cite{kohavi1995study}.
In contrast to standard confidence calculations, there is the added complexity
of imputation taking place at various points in the plan and the composition of results
through standard query operations.

\textbf{Imputation operators:} We explored a subset of possible imputation operators.
In particular, we explored two variants of the \textit{Drop} and \textit{Impute} operations. The system
could be extended to consider a broader family of operators. As the search space grows,
there will likely be additional steps needed in order to avoid sub-optimal plans.

\textbf{Adaptive query planning:} Currently, the imputation is local to a given query, meaning
no information is shared across queries. An intriguing direction would be to take advantage
of imputation and execution data generated by repeated queries. Note that this goes beyond
simply caching imputation results, and instead could entail operations such as extending
intermediate results with prior imputation values to improve accuracy, or pruning out
query plans which have been shown to produce low-confidence imputations.

%%% Local Variables:
%%% mode: latex
%%% TeX-master: "main"
%%% End:
