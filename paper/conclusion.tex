\section{Conclusion}
\todo{jose: rewrite this}
As shown in \ProjectName{}, missing values and their imputation can successfully be integrated into the relational calculus and
existing plan optimization frameworks. We implement imputation actions, such as dropping or imputing values with a machine
learning technique, as operators in the algebra and use a simple, but effective, cost model to consider tradeoffs in 
information loss and time. In effect, user preferences for one or the other can be easily incorporated by adjusting a parameter.
Simple histogram transformations
provide incrementally updated cardinality estimates across operators in the plans, which allow us to provide more accurate cost estimates.
By taking a dynamic programming approach, we can consider a variety of operator placements
and input columns, while keeping planning tractable in real-world examples.

In our experiments, we considered a series of analytic queries (using relevant aggregates) and set queries (with simple projections) on real-world
American Community Survey data and synthetic data
and showed that the chosen query plan improves accuracy of results relative to simply ignoring missing values. Furthermore,
the variety of imputation operator placements across queries further emphasizes the lack of flexibility imposed by the coarse-grained
pre-processing approach to imputation. By allowing different imputation strategies for different queries, we
take a fine-grained approach to missing data, better addressing the often differing needs of database users.

We highlight the long history of dealing with missing data both in the statistical learning and database communities.
Similarly to existing work, we consider the impact of null values in databases and develop a simple set of invariants to 
successfully plan around them. In contrast to the statistical learning work, our emphasis is not on the specific algorithm
used to impute but rather on the timing of imputation in the execution of a query. In contrast to existing database work,
we incorporate imputation into a cost-based optimizer and hide any details
regarding missing values inside the system, allowing users to use traditional SQL and engage in normal workloads.

\subsection{Future Work}
\todo{Jose: rewrite this}
\ProjectName{} opens up multiple avenues for further work. For instance, we
could extend our minimal/maximal imputation operators to consider global
information, such as the specific columns needed in operators higher up in the
query plan.  Our optimizer uses the same machine learning algorithm in all
instances of imputation operators. Integrating multiple possible algorithms
could allow for further fine-grained imputation, and honing the time complexity and
loss of these algorithms for an iterator model database could facilitate more
intelligent query plans and consistent interpretations of $\alpha$ across
strategies. For example, algorithms that learn in an online manner could
increase the efficiency of the system.  Furthermore, a multiple imputation
strategy could be followed for queries involving certain aggregates. Finally,
the underlying database for \ProjectName{} is far from a full-featured,
production database, so a natural step is to integrate imputation planning in a
widely-adopted production-quality database. This will likely expose further
opportunities for improvement.





%%% Local Variables:
%%% mode: latex
%%% TeX-master: "main"
%%% End:
