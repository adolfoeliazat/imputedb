\section{Conclusion}
Our evaluation of \ProjectName{} shows that imputation of missing values can be successfully integrated into the relational algebra and
existing cost-based plan optimization frameworks. We implement imputation actions, such as dropping or imputing values with a machine
learning technique, as operators in the algebra and use a simple yet effective cost model to consider trade-offs in 
imputation quality and runtime. We show that different values for the trade-off parameter can yield substantially
different plans, allowing the user to express their preferences for performance on a per-query basis.
Our experiments with the CDC NHANES, freeCodeCamp and ACS datasets
show that \ProjectName{} can be used successfully with real-world data to run standard SQL queries over data with missing values. 
We craft a series of realistic queries, meant to resemble the kind of questions an analyst might ask during
the data exploration phase. The plans selected for each query execute an order-of-magnitude faster than
 the standard approach of imputing on the entire base tables and
then formulating queries. Furthermore, the difference in query results between the two approaches is
shown to be small in all queries considered (\lowsmapealphazero{}-\highsmapealphaone{}\% error, \Cref{sec:results}). Data analysts need not commit to a specific imputation strategy and can instead
vary this across queries.

We discuss the long history of dealing with missing data both in the statistical learning and database communities.
Similar to existing work, we consider the impact of \nullv{} values in databases and develop a simple set of invariants to 
successfully plan around them. In contrast to existing statistical learning work, our emphasis is not on the specific algorithm
used to impute but rather on the placement of imputation steps in the execution of a query. In contrast to existing database work,
we incorporate imputation into a cost-based optimizer and hide any details
regarding missing values inside the system, allowing users to use traditional SQL as if the database were complete.
While prior work has incorporated operations such as prediction into their databases, this has been domain-specific
in some cases, and in others, operations have not been integrated into the planner. The novelty
of our contribution lies in the formalization of missing value imputation for query planning, which results in performance
gains over traditional approaches. This approach acknowledges that different users performing different queries on the same
dataset will likely have varying imputation needs and that execution plans should appropriately reflect that variation.

\subsection{Future Work}
\ProjectName{} opens up multiple avenues for further work.

\begin{itemize}
\item \textbf{Imputation confidence:} Obtaining confidence measures for query results
produced when missing values are imputed with a statistical model is
another possible improvement to the system. These measures could be obtained in general
using resampling methods like cross-validation and bootstrapping\cite{kohavi1995study}.
In contrast to standard confidence calculations, one added complexity is that
imputation takes place at various points in the plan, so the composition of results
through standard query operations needs to be taken into account when computing confidence
measures.
In addition, multiple imputation could be used within query execution and confidence measures like variances could
similarly propagate through the query plan.

\item \textbf{Imputation operators:} We explored a subset of possible imputation operators.
In particular, we explored two variants of the \textit{Drop} and \textit{Impute} operations. The system
could be extended to consider a broader family of operators. As the search space grows,
there will likely be additional steps needed in order to avoid sub-optimal plans.

\item \textbf{Adaptive query planning:} Currently, the imputation is local to a given query, meaning
no information is shared across queries. An intriguing direction would be to take advantage
of imputation and execution data generated by repeated queries. Note that this goes beyond
simply caching imputation results, and instead could entail operations such as extending
intermediate results with prior imputation values to improve accuracy, or pruning out
query plans which have been shown to produce low-confidence imputations.
\end{itemize}

%%% Local Variables:
%%% mode: latex
%%% TeX-master: "main"
%%% End:
