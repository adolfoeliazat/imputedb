\section{Introduction}

Handling incorrect or dirty data is a complex and challenging problem for analysts.
According to some estimates, data scientists spend between 50 and 80 percent of their
time performing tasks such as collecting and cleaning datasets\cite{data-science-cleaning}.
One way in which a dataset can be dirty is for parts of it to be missing altogether.
Missing data is one of the simpler variants of dirty data, but if handled naively, can still cause analyses to be incorrect.
To handle this problem, users may manually clean their dataset by performing
some statistical analysis to replace missing data elements with likely values.
This process is called \emph{imputation}.
Traditional imputation methods replace all missing values in a dataset to create a new clean dataset.
Although manual imputation solves the problem of missing data, in the age of big data it may be very expensive to run an imputation algorithm on an entire dataset.
Additionally, it may not even be necessary to completely clean the data to make it usable.
Some users may be willing to run queries on dirty data, simply ignoring the missing values, as long as they do not have to pay the cost of imputation.
Others may want to run queries on a subset of the data, and so do not need to impute
every field in every record. Yet others may want to customize the 
imputation algorithm for the tradeoffs and demands of a particular domain.

In this paper, we present \ProjectName{}, a database system which is designed to interact with a dirty dataset as though it were clean.
The guiding design principle behind \ProjectName{} is that the user should never see missing data or have to modify their queries to account for it.
To achieve this goal, we perform imputation on the fly, during query execution.
Performing imputation at query time allows our system to impute only the data necessary to run the query, and it allows users to flexibly trade accuracy for computation time.

%%% Local Variables:
%%% mode: latex
%%% TeX-master: "main"
%%% End:
